\documentclass[11pt]{article}

\title{Model Checking World Domination}
\author{Jacob Errington \& Kevin Li}
\date{Formal Verification -- COMP 525\\18 April 2017}

\usepackage{tikz}
\usepackage{amsthm,amsmath,amssymb}
\usepackage{csquotes}
\usepackage[margin=2.0cm]{geometry}
\usetikzlibrary{automata}
\usetikzlibrary{graphs}

\theoremstyle{definition}
\newtheorem{defn}{Definition}

\begin{document}

\maketitle

\begin{abstract}
    Probabilistic model checking is employed
    in swarm robotics to verify
    properties of swarm behavior in the context of
    tasks. Several canonical tasks are often used
    in testing swarms for desired collective behavior,
    such as foraging, flocking, and navigation.
    Swarms performing such tasks face numerous challenges,
    one of which is task allocation, which is concerned 
    with optimal division of labour between individuals 
    and groups of robots in a swarm.  Extant literature 
    on robotic swarm task allocation focuses on the 
    aforementioned tasks. In this paper we introduce the
    notion of distributable probabilistic tasks,
    a class of tasks with particular traits, and
    propose a probabilistic model checking framework
    through which optimal task allocation proportions
    can be derived. We then conduct experiments using our
    framework in the context of several combat scenarios,
    and present experimental results.
\end{abstract}

\section{Introduction}\label{sec:intro}

Formal verification, and in particular probabilistic model
checking, has played a major role in the engineering of
correct swarm and individual robot behavior. % Citation needed.

In section \ref{sec:lit-review} we provide an overview of the
existing literature on probabilistic model checking and
task allocation in swarm robotics. In section \ref{sec:dpts}
we describe the problem domain that our framework tackles,
along with our definition of a distributable probabilistic
task. In section \ref{sec:model} we describe how our framework
is constructed using a concrete combat example. In section
\ref{sec:implementation} we describe the implementation of
our framework, along with the specifications of how our
experiments were implemented. In section \ref{sec:results}
we present our experimental results using our framework.

\section{Literature Review}\label{sec:lit-review}

\section{Distributable Probabilistic Tasks}\label{sec:dpts}

\section{Model Construction}\label{sec:model}

\section{Experiment Implementation}\label{sec:implementation}

\section{Experimental Results}\label{sec:results}

\section{References}

\begin{thebibliography}{99}
\end{thebibliography}

\end{document}
