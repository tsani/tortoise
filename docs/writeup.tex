\documentclass[11pt]{article}

\title{Model Checking World Domination}
\author{Jacob Errington \& Kevin Li}
\date{Formal Verification -- COMP 525\\18 April 2017}

\usepackage{tikz}
\usepackage{amsthm,amsmath,amssymb}
\usepackage{csquotes}
\usepackage[margin=2.0cm]{geometry}
\usetikzlibrary{automata}
\usetikzlibrary{graphs}

\theoremstyle{definition}
\newtheorem{defn}{Definition}

\begin{document}

\maketitle

\begin{abstract}
    Probabilistic model checking is employed
    in swarm robotics to verify
    properties of swarm behavior in the context of
    tasks. Several canonical tasks are often used
    in testing swarms for desired collective behavior,
    such as foraging, flocking, and navigation.
    Swarms performing such tasks face numerous challenges,
    one of which is task allocation, which is concerned
    with optimal division of labour between individuals
    and groups of robots in a swarm.  Extant literature
    on robotic swarm task allocation focuses on the
    aforementioned tasks. In this paper we introduce the
    notion of distributable probabilistic tasks,
    a class of tasks with particular traits, and
    propose a probabilistic model checking framework
    through which optimal task allocation proportions
    can be derived. We then conduct experiments using our
    framework in the context of several combat scenarios,
    and present experimental results.
\end{abstract}

\section{Introduction}\label{sec:intro}

A robot swarm is a collection of robots (that are often
identical) that act cooperatively to perform a task.
The scope of this paper does not cover the (daunting) challenges
involved in the engineering of a robot swarm, nor does
it take practical considerations such as communication between
robots or hardware. In this paper we will take an abstract view
of a robotic swarm and assume (perhaps recklessly) that
these issues have all been taken care of.

Formal verification, and in particular probabilistic model
checking, has played a major role in the engineering of
correct swarm and individual robot behavior. % Citation needed.

There are generally two approaches for modeling a swarm robotics
system: the microscopic and macroscopic view.

In the microscopic view, the behavior of an individual robot
is first modeled with a transition system. Then,
a model corresponding to the behavior of the swarm is
generated by creating a composed transition system
by composing many of the individual transition systems
together. This representation is useful for verifying
properties of individual robots, but is not
very useful for the verification of properties of the
entire swarm when the population of robots in the swarm
is high, as the state space grows exponentially in
the number of composed transition systems.

In the macroscopic view (assuming that the robots
in the swarm are identical in design and can be
modeled with the same transition system), a single
transition system is used to represent the entire
swarm. The transition system is identical to the
transition system that models individual behavior of
a single robot, except where each state is
annotated with a variable that describes the
number of robots in the swarm that are currently
in that state. \cite{konur12}

By representing a swarm robotics system this way,
we can describe a far larger swarm,
which makes the properties that can be verified
much more interesting. Thus, this is the representation
of a robot swarm that we will use for this paper.

In section \ref{sec:background-motivation} we provide an overview of the
existing literature on probabilistic model checking and
task allocation in swarm robotics. We then introduce
the problem domain that our framework tackles,
along with our definition of a distributable probabilistic
task. In section \ref{sec:model} we describe how our framework
is constructed using a concrete combat example. In section
\ref{sec:implementation} we describe the implementation of
our framework, along with the specifications of how our
experiments were implemented. In section \ref{sec:results}
we present our experimental results using our framework.
In section \ref{sec:conclusion} we add some concluding remarks
and areas for future improvement.

\section{Background and Motivation}\label{sec:background-motivation}

The real-world results of actions are uncertain. Thus,
probabilistic models are ideal for representing
real-world swarm robotics systems.
In particular, discrete time Markov chains (DTMCs)
are used to extend the transition system representation
of a robot swarm, to encode the effects of taking actions
that produce probabilistic results.

Liu, Winfield, and Sa represent a foraging swarm
using a discrete time Markov chain. Figure \ref{fig:foraging}
shows their construction using the macroscopic
representation of a robot swarm.

\begin{figure}
    \caption{Konur, Dixon, and Fisher's foraging swarm representation}
    \label{fig:foraging}
    \centering
    \includegraphics[width=\textwidth]{foraging.png}

    The macroscopic representation of a robotic swarm
    shows that each robot is responsible for attempting
    to gather "food" in order to increase the overall
    energy levels of the entire swarm. The semantics
    of the states are:

    \begin{description}
        \item[Searching] Robot is searching for food
        \item[Grabbing] Robot attempts to grab a food item
        \item[Depositing] Robot returns home with the food item
        \item[Homing] Robot returns home without any food items
        \item[Resting] Robot rests for some time
    \end{description}

    Then, there are some timeout conditions and probability
    distributions over states after a transition is performed.

    \begin{itemize}
        \item $ T_s $ is the maximum amount time a robot can continue searching.
        \item $ T_g $ is the maximum amount of time a robot can attempt
            grabbing
        \item $ T_d = \frac{T_h}{T_r} $ is the average time spent depositing
        \item $ \gamma_f $ is the probability of finding a food item
        \item $ \gamma_g $ is the probability of grabbing a food item
        \item $ \gamma_l $ is the probability of losing sight of a food
            item.
    \end{itemize}

    After including timeout conditions in the guards
    of transitions, the model as introduced by Konur, Dixon, and
    Fisher is actually a probabilistic timed automaton. However,
    ignoring the guards on clock values and removing all self-transitions
    with only clock guards on them and clock guards, we can consider this model
    exactly as a discrete time Markov chain. Then, we can see
    that a robot in the "Searching" state can probabilistically
    transition into either the "Grabbing" state, or continue to
    search for food items.

    Side effects are then added. Total swarm energy levels are increased
    with every unit of food deposited at home. Energy levels decrease
    steadily when actions are taken by individual
    robots. Now this model can then be used as a transition system in
    which properties can be verified.
\end{figure}

After constructing such a model, they implemented it in the PRISM model checker
(PRISM) to verify actual properties in this probabilistic domain, such
as the probability that the total energy supply of the swarm
falls below some level.

The properties are written in Probabilistic Computation Tree
Logic (PCTL) \cite{pctl}, which extend Computation Tree Logic (CTL)
with the following syntax and semantics.

\begin{align*}
    s \vDash & \mathcal{P}_{\sim\lambda}(\phi_1 \text{ U } \phi_2) \\
    s \vDash & \mathcal{P}_{\sim\lambda}(\square \phi) \\
    \text{where } & \sim \in \{ >, <, \leq, \geq \} \\
                  & \phi_i \text{ is a CTL formula.} \\
                  & \lambda \text{ is a probability.} \\
\end{align*}

The semantics of $ s \vDash \mathcal{P}_{\geq\lambda} ( \phi ) $ are
"s satisfies $ \mathcal{P}_{\geq\lambda} ( \phi ) $ iff s
satisfies $ \phi $ with probability greater than or equal
to $ \lambda $".
The $ >, <, \leq $ cases for $ \sim $ follow similarly.

PCTL properties can be written directly into PRISM, and
given a model that supports the property (i.e. there is
no variable in the property that is not mentioned
in the model) such a property can be verified.
PRISM supports an additional feature: if the $ \lambda $
parameter is not specified in the property, then
instead saying if the property is satisfied or not
satisfied, PRISM will return the probability that
the property is specified.

\section{Model Construction}\label{sec:model}

\section{Experiment Implementation}\label{sec:implementation}

\section{Experimental Results}\label{sec:results}

\section{Conclusion}\label{sec:conclusion}

\section{References}

\begin{thebibliography}{99}
    \bibitem{konur12}
        S. Konur, C. Dixon, and M. Fisher.
        Analysing robot swarm behaviour via
        probabilistic model checking.
        Robotics and Autonomous Systems, 60:199–213,
        2012.
    \bibitem{konur10}
        Savas Konur , Clare Dixon , Michael Fisher.
        Formal verification of probabilistic swarm behaviours,
        Proceedings of the 7th international conference on Swarm intelligence,
        September 08-10, 2010, Brussels, Belgium
    \bibitem{foraging}
        Liu, W., Winfield, A., Sa, J.: Modelling Swarm Robotic Systems: A Study in Collective
        Foraging. In: Proc. Towards Autonomous Robotic Systems (TAROS). pp. 25–32 (2007)
    \bibitem{pctl}
        Hansson, Hans, and Bengt Jonsson. "A logic for reasoning about time and reliability." Formal aspects of computing 6.5 (1994): 512-535.
\end{thebibliography}

\end{document}
