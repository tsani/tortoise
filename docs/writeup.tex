\documentclass[11pt]{article}

\title{Model Checking World Domination}
\author{Jacob Errington \& Kevin Li}
\date{Formal Verification -- COMP 525\\18 April 2017}

\usepackage{tikz}
\usepackage{amsthm,amsmath,amssymb}
\usepackage{csquotes}
\usepackage[margin=2.0cm]{geometry}
\usetikzlibrary{automata}
\usetikzlibrary{graphs}

\theoremstyle{definition}
\newtheorem{defn}{Definition}

\begin{document}

\maketitle

\begin{abstract}
    Probabilistic model checking is employed
    in swarm robotics to verify
    properties of swarm behavior in the context of
    tasks. Several canonical tasks are often used
    in testing swarms for desired collective behavior,
    such as foraging, flocking, and navigation.
    Swarms performing such tasks face numerous challenges,
    one of which is task allocation, which is concerned 
    with optimal division of labour between individuals 
    and groups of robots in a swarm.  Extant literature 
    on robotic swarm task allocation focuses on the 
    aforementioned tasks. In this paper we introduce the
    notion of distributable probabilistic tasks,
    a class of tasks with particular traits, and
    propose a probabilistic model checking framework
    through which optimal task allocation proportions
    can be derived. We then conduct experiments using our
    framework in the context of several combat scenarios,
    and present experimental results.
\end{abstract}

\section{Introduction}\label{sec:intro}

Formal verification, and in particular probabilistic model
checking, has played a major role in the engineering of
correct swarm and individual robot behavior. % Citation needed.

There are generally two approaches for modeling a swarm robotics
system: the microscopic and macroscopic view.

In the microscopic view, the behavior of an individual robot
is first modeled with a transition system. Then,
a model corresponding to the behavior of the swarm is
generated by creating a composed transition system
by composing many of the individual transition systems
together. This representation is useful for verifying
properties of individual robots, but is not
very useful for the verification of properties of the
entire swarm when the population of robots in the swarm
is high, as the state space grows exponentially in
the number of composed transition systems.

In the macroscopic view (assuming that the robots
in the swarm are identical in design and can be
modeled with the same transition system), a single
transition system is used to represent the entire
swarm. The transition system is identical to the
transition system that models individual behavior of
a single robot, except where each state is
annotated with a variable that describes the
number of robots in the swarm that are currently
in that state. \cite{konur12}

By representing a swarm robotics system this way,
we can describe a far larger swarm,
which makes the properties that can be verified
much more interesting. Thus, this is the representation
of a robot swarm that we will use for this paper.

In section \ref{sec:background-motivation} we provide an overview of the
existing literature on probabilistic model checking and
task allocation in swarm robotics. We then introduce
the problem domain that our framework tackles,
along with our definition of a distributable probabilistic
task. In section \ref{sec:model} we describe how our framework
is constructed using a concrete combat example. In section
\ref{sec:implementation} we describe the implementation of
our framework, along with the specifications of how our
experiments were implemented. In section \ref{sec:results}
we present our experimental results using our framework.
In section \ref{sec:conclusion} we add some concluding remarks
and areas for future improvement.

\section{Background and Motivation}\label{sec:background-motivation}


\section{Model Construction}\label{sec:model}

\section{Experiment Implementation}\label{sec:implementation}

\section{Experimental Results}\label{sec:results}

\section{Conclusion}\label{sec:conclusion}

\section{References}

\begin{thebibliography}{99}
    \bibitem{konur12}
        S. Konur, C. Dixon, and M. Fisher.
        Analysing robot swarm behaviour via
        probabilistic model checking.
        Robotics and Autonomous Systems, 60:199–213,
        2012.
\end{thebibliography}

\end{document}
